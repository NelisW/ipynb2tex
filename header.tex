% the first cell of the notebook must be a RawNBConvert cell with the following contents:
%
%%%\documentclass that you want to use
%%%\documentclass{report}

\usepackage{listings}
\usepackage{color}
\usepackage{graphicx}
\usepackage[a4paper, margin=0.75in]{geometry}
\usepackage{textcomp} % additional fonts, required for upquote in listings and \textmu
\usepackage{placeins} % FloatBarrier
\usepackage{url} % for websites
\usepackage[detect-weight]{siunitx} % nice! SI units and print numbers
\usepackage{afterpage} % afterpage{\clearpage}
\usepackage{gensymb} % get the degree symbol as in \celcius
\usepackage{amsmath} 

%the following is required for carriage return symbol
%ftp://ftp.botik.ru/rented/znamensk/CTAN/fonts/mathabx/texinputs/mathabx.dcl
%https://secure.kitserve.org.uk/content/mathabx-font-symbol-redefinition-clash-latex
\DeclareFontFamily{U}{mathb}{\hyphenchar\font45}
\DeclareFontShape{U}{mathb}{m}{n}{
      <5> <6> <7> <8> <9> <10> gen * mathb
      <10.95> mathb10 <12> <14.4> <17.28> <20.74> <24.88> mathb12
      }{}
\DeclareSymbolFont{mathb}{U}{mathb}{m}{n}
\DeclareMathSymbol{\dlsh}{3}{mathb}{"EA}

\usepackage[T1]{fontenc}

\definecolor{LightGrey}{rgb}{0.95,0.95,0.95}
\definecolor{LightRed}{rgb}{1.0,0.9,0.9}

\lstset{ %
upquote=true, % gives the upquote instead of the curly quote
basicstyle=\ttfamily\footnotesize,       % the size of the fonts that are used for the code
numbers=none,                   % where to put the line-numbers
showspaces=false,               % show spaces adding particular underscores
showstringspaces=false,         % underline spaces within strings
showtabs=false,                 % show tabs within strings adding particular underscores
frame=lines,                   % adds a frame around the code
tabsize=4,              % sets default tabsize to 2 spaces
captionpos=b,                   % sets the caption-position to bottom
framesep=1pt,
xleftmargin=0pt,
xrightmargin=0pt,
 captionpos=t,                    % sets the caption-position to top
%deletekeywords={...},            % if you want to delete keywords from the given language
%escapeinside={\%*}{*)},          % if you want to add LaTeX within your code
%escapeinside={\%}{)},          % if you want to add a comment within your code
breaklines=true,        % sets automatic line breaking
breakatwhitespace=false,    % sets if automatic breaks should only happen at whitespace
prebreak=\raisebox{0ex}[0ex][0ex]{$\dlsh$} % add linebreak symbol
}

\lstdefinestyle{incellstyle}{
  backgroundcolor=\color{LightGrey},  % choose the background color,  add \usepackage{color}
  language=Python,
}

\lstdefinestyle{outcellstyle}{
  backgroundcolor=\color{LightRed},   % choose the background color; you must add \usepackage{color} or \usepackage{xcolor}
}

\setlength{\parindent}{0.0mm}
\setlength{\parskip}{6pt}

\begin{document}


%%%whatever preamble lines you require
%%%\begin{document}
%%%whatever extra info before the notebook follows
%%%\end{document} % is added by the converter.
%
%The header.tex file provides the functionality required by the converter-created code. 
%This content is input using the \LaTeX{} input command.
%
%
% - The exporter extracts the document class line from the first cell of the notebook.
% - Then follows the contents of this file (as preamble)
% - Then follows the rest of the first cell of the notebook (rest of preamble, begin doc, etc)
% - Finally the rest of the notebook is exported.
%


\usepackage{ragged2e}
\usepackage{listings}
\usepackage{color}
\usepackage{graphicx}
\usepackage{textcomp} % additional fonts, required for upquote in listings and \textmu
\usepackage{placeins} % FloatBarrier
\usepackage{url} % for websites
\usepackage[detect-weight]{siunitx} % nice! SI units and print numbers
\usepackage{afterpage} % afterpage{\clearpage}
\usepackage{gensymb} % get the degree symbol as in \celcius
\usepackage{amsmath} 
\usepackage[printonlyused]{acronym}
\usepackage{lastpage}

%the following is required for carriage return symbol
%ftp://ftp.botik.ru/rented/znamensk/CTAN/fonts/mathabx/texinputs/mathabx.dcl
%https://secure.kitserve.org.uk/content/mathabx-font-symbol-redefinition-clash-latex
\DeclareFontFamily{U}{mathb}{\hyphenchar\font45}
\DeclareFontShape{U}{mathb}{m}{n}{
      <5> <6> <7> <8> <9> <10> gen * mathb
      <10.95> mathb10 <12> <14.4> <17.28> <20.74> <24.88> mathb12
      }{}
\DeclareSymbolFont{mathb}{U}{mathb}{m}{n}
\DeclareMathSymbol{\dlsh}{3}{mathb}{"EA}

\usepackage[T1]{fontenc}

\definecolor{LightGrey}{rgb}{0.95,0.95,0.95}
\definecolor{LightRed}{rgb}{1.0,0.9,0.9}

\lstset{ %
upquote=true, % gives the upquote instead of the curly quote
basicstyle=\ttfamily\footnotesize,       % the size of the fonts that are used for the code
numbers=none,                   % where to put the line-numbers
showspaces=false,               % show spaces adding particular underscores
showstringspaces=false,         % underline spaces within strings
showtabs=false,                 % show tabs within strings adding particular underscores
frame=lines,                   % adds a frame around the code
tabsize=4,              % sets default tabsize to 2 spaces
captionpos=b,                   % sets the caption-position to bottom
framesep=1pt,
xleftmargin=0pt,
xrightmargin=0pt,
 captionpos=t,                    % sets the caption-position to top
%deletekeywords={...},            % if you want to delete keywords from the given language
%escapeinside={\%*}{*)},          % if you want to add LaTeX within your code
%escapeinside={\%}{)},          % if you want to add a comment within your code
breaklines=true,        % sets automatic line breaking
breakatwhitespace=false,    % sets if automatic breaks should only happen at whitespace
prebreak=\raisebox{0ex}[0ex][0ex]{$\dlsh$} % add linebreak symbol
}

\lstdefinestyle{incellstyle}{
  backgroundcolor=\color{LightGrey},  % choose the background color,  add \usepackage{color}
  language=Python,
}

\lstdefinestyle{outcellstyle}{
  backgroundcolor=\color{LightRed},   % choose the background color; you must add \usepackage{color} or \usepackage{xcolor}
}

\usepackage[a4paper, margin=0.75in]{geometry}
\newlength{\textwidthm}
\setlength{\textwidthm}{\textwidth}



